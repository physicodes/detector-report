\chapter{Conclusion}

This low cost proportional counter has been shown to be capable of detecting $\gamma$-ray radiation with energies from a few keV up to tens of keV. It is expected that sources with lower energy radiation will see a sharp drop in detector efficiency as they cannot penetrate the detector wall, though this could be resolved by installing a Mylar window. Higher energy photons will experience a gradual decrease in efficiency as Argon absorption falls away logarithmically \cite{nist}.

Its proportionality and an impressive agreement with the theoretical Diethorn formula is evidence that despite the simplicity and low cost of production, the detector is still good enough to follow the same principles of operation as far more expensive detectors. The detector is capable of measuring the relative intensities and energies of sources and can therefore be used to measure the energy spectrum of sources once it has been calibrated.

The implementation of a kit built preamplifier resulted in an increase of energy resolution of only 37.8\%. However, despite the relative cost and ease of assembly, this setup still measured a recognisable \textsuperscript{55}Fe spectrum which could have been modelled and further analysed if needed.